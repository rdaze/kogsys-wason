\documentclass[11pt,a4paper]{article}

\usepackage[margin=2.5cm]{geometry}
\usepackage{hyperref}
\usepackage{amsmath}
\usepackage{booktabs}

\usepackage[skip=6pt,indent=0pt]{parskip}

\usepackage{enumitem}
\setlist{nosep}

\title{Wason Selection Task Experiment: Evaluation}
\author{Cognitive Systems Project}
\date{January 26, 2026}

\begin{document}

\maketitle

\section{Statistical Evaluation Plan}

\subsection{Overview of Design}
The experiment uses a between-subjects, single-factor design with one trial per participant. Participants are randomly assigned to one of three framing conditions:
\begin{enumerate}
    \item \textbf{Abstract context (A)}: numbers and colors
    \item \textbf{Familiar real-world context (B)}: age and alcohol
    \item \textbf{Unfamiliar structured context (C)}: rule-based access/permission scenario
\end{enumerate}

The logical structure of the Wason Selection Task (WST) is identical across conditions; only the semantic framing differs. Dependent variables are collected per participant, including correctness, response time, selection behavior, and confidence.

\subsection{Primary Hypothesis}
The primary hypothesis concerns performance differences across contextual framing conditions:
\begin{equation}
    \text{Accuracy}_B > \text{Accuracy}_C > \text{Accuracy}_A
\end{equation}
The most important planned comparison is expected to be the difference between the familiar and abstract conditions:
\begin{equation}
    \text{Accuracy}_B > \text{Accuracy}_A
\end{equation}
Differences between familiar and unfamiliar structured contexts are treated as exploratory, conditional on whether the unfamiliar structured scenario supports permission-based reasoning.

\subsection{Ideal Inferential Tests (Final Analysis)}
\subsubsection{Correctness (Primary DV)}
Correctness is binary (correct vs.\ incorrect). The most appropriate final analysis is logistic regression:
\begin{equation}
    \text{Correct} \sim \text{Condition}
\end{equation}

This model yields an omnibus test of whether correctness differs across the three conditions. The primary effect of interest is tested using planned contrasts:
\begin{itemize}
    \item \textbf{Primary planned contrast:} Familiar (B) vs.\ Abstract (A)
    \item \textbf{Secondary contrasts (exploratory):} Familiar (B) vs.\ Unfamiliar structured (C), and Unfamiliar structured (C) vs.\ Abstract (A)
\end{itemize}

Effect sizes will be reported as odds ratios (OR) with 95\% confidence intervals. In the presence of sparse data or complete separation, penalized methods (e.g., Firth logistic regression) will be explored.

Emergency continuity plan: Fisher’s exact tests (pairwise) when expected counts are small.

\paragraph{Effect Size Reporting for Accuracy.}
Multiple effect sizes are appropriate:
\begin{itemize}
    \item \textbf{Difference in proportions:} $\Delta p = p_1 - p_2$
    \item \textbf{Odds ratio:} Odds ratios (OR) from logistic regression
\end{itemize}

\subsubsection{Time to Submission}
Completion time is continuous and typically right-skewed in browser experiments. The ideal analysis uses either:
\begin{itemize}
    \item linear regression / ANOVA on log-transformed time,
          \begin{equation}
              \log(\text{Time}) \sim \text{Condition}
          \end{equation}
\end{itemize}

\subsubsection{Selection Changes}
Selection changes are non-negative integer counts. A generalized linear model should be suitable:
\begin{equation}
    \text{Changes} \sim \text{Condition}
\end{equation}
using a Poisson model.

\subsubsection{First Card Selected}
The first selected card is categorical. A contingency table analysis (Condition $\times$ FirstCard) is appropriate:
\begin{itemize}
    \item chi-square test for independence
\end{itemize}

\subsubsection{Confidence Ratings}
Confidence is recorded on a 0--100 scale. The ideal analysis is linear regression / ANOVA:
\begin{equation}
    \text{Confidence} \sim \text{Condition}
\end{equation}
Because confidence may be strongly related to correctness, an additional model can be informative:
\begin{equation}
    \text{Confidence} \sim \text{Condition} + \text{Correct}
\end{equation}

\newpage

\section{Current Status: Preliminary Results (Pilot Data)}

\subsection{Dataset Summary}
At the time of this interim analysis, the database contains 12 valid participant sessions distributed as follows:
\begin{itemize}
    \item Abstract context (A): $n=4$
    \item Familiar context (B): $n=6$
    \item Unfamiliar structured context (C): $n=2$
\end{itemize}
Group sizes are notably imbalanced and the unfamiliar structured condition has very low $n$, making inferential statistics unreliable at this stage. Therefore, current results are interpreted descriptively.

\subsection{Accuracy by Condition}
Observed correctness rates are:
\begin{itemize}
    \item Abstract (A): $2/4 = 50\%$
    \item Familiar (B): $3/6 = 50\%$
    \item Unfamiliar structured (C): $0/2 = 0\%$
\end{itemize}

Zuerst Deskriptiv. e.g. Balken diagramm
Deskriptiv verschaulichen wie viele sind auf Modus Tollens, pollens, yada yada drauf eingefallen


\end{document}


