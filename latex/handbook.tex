\documentclass[11pt,a4paper]{article}

\usepackage[margin=2.5cm]{geometry}
\usepackage{hyperref}
\usepackage{amsmath}
\usepackage{booktabs}

\usepackage[skip=6pt,indent=0pt]{parskip}

\usepackage{enumitem}
\setlist{nosep}

\title{Wason Selection Task Experiment: Technical Documentation}
\author{Cognitive Systems Project}
\date{January 8, 2026}

\begin{document}

\maketitle
\tableofcontents
\newpage

% =========================================================
\section{Experiment Overview}

The experiment implements a single-trial, browser-based version of the Wason Selection Task to study the effect of contextual framing on conditional reasoning.

Participants evaluate a conditional rule of the form \emph{If P, then Q} by selecting which of four cards must be turned over to test the rule.

The logical structure is held constant across conditions; only the semantic context varies.

Three conditions are used:
\begin{itemize}
  \item abstract (numbers and colors)
  \item familiar real-world (age and alcohol)
  \item unfamiliar structured (rule-based access scenario)
\end{itemize}

Participants are randomly assigned to exactly one condition.

% =========================================================
\section{Experimental Design}

\subsection{Design}

Between-subjects, single-factor design with one trial per participant.

\subsection{Independent Variable}

Context type with three levels:
\begin{itemize}
  \item abstract
  \item familiar
  \item unfamiliar structured
\end{itemize}

Assignment is uniform random.

\subsection{Dependent Variables}

Recorded per participant:
\begin{itemize}
  \item \textbf{Correctness}: exact match between final selection and condition-specific correct set
  \item \textbf{Time to submission}: task submit time minus task start time (ms)
  \item \textbf{Selection changes}: number of card toggle actions
  \item \textbf{First card selected}: identifier of first toggled card
  \item \textbf{Confidence}: self-report, 0--100
\end{itemize}

\subsection{Participant Flow}

Introduction $\rightarrow$ task $\rightarrow$ confidence rating $\rightarrow$ debrief.  
Backward navigation is not permitted.

% =========================================================
\section{Front-End Architecture}

\subsection{Stack}

Client-side application using:
\begin{itemize}
  \item Vite
  \item React
  \item TypeScript
  \item Tailwind CSS
\end{itemize}

The application is deployed as a static website.

\subsection{State Model}

A single global session state controls the application.  
A discrete screen variable determines the rendered view:
\begin{itemize}
  \item \texttt{start}
  \item \texttt{task}
  \item \texttt{grade}
  \item \texttt{end}
\end{itemize}

Transitions are strictly forward-only.

\subsection{Component Structure}

\begin{itemize}
  \item \texttt{App.tsx}: global state and transitions.
  \item \texttt{screens/}: UI per experimental stage.
  \item \texttt{tasks.ts}: condition definitions and correct sets.
  \item \texttt{types.ts}: shared type definitions.
  \item \texttt{api.ts}, \texttt{supabase.ts}: authentication and persistence.
\end{itemize}

Screen components are stateless with respect to experimental logic.

% =========================================================
\section{Task Logic and Measurement}

\subsection{Card Model}

Each task consists of four cards identified as \texttt{c1--c4}.  
Identifiers are constant; labels vary by condition.

\subsection{Selection Logic}

Card interaction is implemented as a toggle:
\begin{itemize}
  \item selecting adds the card to the selection set
  \item selecting again removes it
\end{itemize}

All toggles are centrally handled.

\subsection{First Card Selected}

The first toggle event sets \texttt{first\_card\_selected}.  
This value is immutable after initialization.

\subsection{Selection Changes}

Each toggle increments \texttt{selection\_changes}.  
This counts exploratory behavior independently of correctness.

\subsection{Final Selection}

The final selection is the unordered set of cards selected at submission.

\subsection{Correctness}

A response is correct iff:
\begin{itemize}
  \item all required cards are selected
  \item no additional cards are selected
\end{itemize}

Partial or superset selections are scored as incorrect.

\subsection{Timing}

Task duration is measured using \texttt{performance.now()}:
\[
\text{time} = \text{task\_submit\_ms} - \text{task\_start\_ms}
\]

\subsection{Confidence}

Confidence is collected post-task on a 0--100 scale.

% =========================================================
\section{Back-End and Data Storage}

\subsection{Infrastructure}

Data are stored using Supabase (PostgreSQL).  
No custom server is used.

\subsection{Authentication}

Participants are authenticated via anonymous Supabase auth.  
The resulting UUID serves as the participant identifier.

No personally identifying data are collected.

\subsection{Security Model}

Row Level Security enforces:
\begin{itemize}
  \item insert-only access
  \item authenticated users only
  \item user may insert only their own rows
  \item no client-side reads, updates, or deletes
\end{itemize}

The public API key does not grant privileged access.

\subsection{Write Procedure}

Data are written exactly once, after confidence submission.  
A saving state prevents duplicate inserts.  
Errors are surfaced to the participant and allow retry.

% =========================================================
\section{Data Schema}

Each row represents one completed session.

\subsection{Core Fields}

\begin{itemize}
  \item \texttt{session\_id}: client-generated primary key
  \item \texttt{experiment\_id}: experiment/version label
  \item \texttt{user\_id}: anonymous Supabase UUID
  \item \texttt{condition}: assigned context
  \item \texttt{task\_start\_ms}, \texttt{task\_submit\_ms}
  \item \texttt{selection\_changes}
  \item \texttt{first\_card\_selected}
  \item \texttt{final\_selection}
  \item \texttt{correct}
  \item \texttt{confidence}
  \item \texttt{created\_at}
\end{itemize}

\subsection{Derived Measures}

\begin{itemize}
  \item time to submission
  \item exploratory intensity
  \item initial reasoning strategy
\end{itemize}

\subsection{Data Handling}

The database is immutable from the client.  
All exclusions or corrections are performed offline.

% =========================================================
\section{Conclusion}

This document specifies the complete experimental logic, measurement definitions, and data guarantees of the implemented Wason Selection Task system.

\end{document}
